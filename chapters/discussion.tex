% !TEX root = ../root.tex

\chapter{Discussion}\label{chap:discussion}
% Results and discussion, 15-25 pages

For a detailed discussion specific to the results of each paper, we refer to the papers themselves. Here we will discuss the common themes, the connections between them, and what the results of this thesis mean for future work.

\section{Designing robots that can avoid failure}

As demonstrated in~\cite{heim2018shaping}, robustness to failures is critical to allowing learning control, and appropriate system design can greatly influence the inherent system robustness. Our work in~\cite{heim2019beyond} formalizes what a design should strive for: robust natural dynamics. \par
I show an example of computational design in~\cite{heim2019beyond}. However, it is limited to optimizing the parameters of a low-level control for a relatively simple legged system. To truly make use of these concepts for computational design (both of robot hardware and low-level controls) will require further progress in scaling the tools to compute viable sets in state-action space. \par
The work in~\cite{heim2019learnable} begins to address this: although the main contribution of the paper is to enable safe learning directly on a robot, it can also be used in simulation to learn approximations of viable sets, and scales better than the brute-force computation used in~\cite{heim2019beyond}.
Nonetheless, further tools to allow computational tractability are needed to make these concepts useful for higher-dimensional systems. Exploiting structure in well-understood dynamical systems, such as the rigid-body dynamics commonly seen in robots, is a promising path. How this structure relates to the set-valued dynamics view of viability is an open question. \par
In parallel to developing computational tools, I also see value in the intuition gleaned from reduced order models. Indeed, armed with the knowledge that robustness to noisy action spaces will make controller design easier, a robot designer can already make more deliberate design decisions, guided by this intuition.

\section{Designing robots that can fail}
The work of this thesis also emphasizes the need for robots to be designed such that occasional failures are not critical. The most important reason for this is that guaranteeing safety requires perfect models, which, of course, do not exist~\cite{box1976science}. The computational tractability of sufficiently accurate models is, as discussed above, still a major challenge. It is then only reasonable to begin with simpler, less accurate but more tractable models, before transferring directly to the real system. To use these less accurate results in practice, we can take two approaches. \par
One approach is to compute under-approximated viable sets, which will retain guarantees of safe operation. However, these tools also difficult to scale to higher dimensions, and often result in over-conservative approximations~\cite{manchester2011regions}. \par
A different approach, which is mentioned in~\cite{heim2019learnable}, is to use data from actual operation of the robot to refine an initial simulation-driven approximation of the viable set. This approach is contingent on robots being allowed to fail occasionally. In practice, this means a robot should be sturdy enough to survive failures, but also that a failure does not cause unacceptable damage to the user or environment, and that it is possible to reset the robot quickly and easily. Just as in curriculum learning, it may be possible to gather at least a portion of these data samples in a controlled environment, a practice room, so to speak, where failures are tolerable. \par

A second reason why robots should be designed to allow failures is illustrated in~\cite{heim2018unviable}. The core concept is that systems often have large sets of unviable states: states which have not failed yet, but are doomed to fail within finite time. While marching towards failure, the learning agent can still explore and collect informative samples that are useful for the learning process - if it can survive the failure. For this to be useful, two requirements should be fulfilled: first, information relevant to the learning process should be available in unviable states. This will depend largely on the reward function. Second, a well-designed system will not fail abruptly, but rather be able to continue exploration for as long as possible, delaying the inevitable failure. Fortunately, this requirement often goes hand in hand with improving robustness, as discussed above. For example, starting with soft gains on an impedance controller will allow a robot to stumble and fall more slowly than a controller that is aggressively tuned for performance.

\section{Leveraging dynamics models}
I have focused on model-free methods, for a very deliberate reason: I believe that model-based optimal control is a fantastically powerful tool, and it is too tempting to try to reformulate any problem as an optimization problem.
At the same time, I believe that robustness is of paramount importance and has been overshadowed by the optimality perspective of dynamics.
Taking a completely (or nearly) model-free approach forces us to deal directly with robustness, as we have no assumed structure to exploit.
It has strongly motivated the development of the new, mathematically simple objects presented in~\cite{heim2019beyond}, and the clean simplicity for safe learning in~\cite{heim2019learnable}.
% For example, contemporary work on safe learning based on viability make use of relatively sophisticated mathematical tools such as differential inclusions AUBIN, barrier functions EGERSTEDT-AMES, or solving Hamilton-Jacobi-Issacs equations TOMLIN.
Now that the fundamental mathematical objects have been established, the time is ripe for turning to model-based tools. Leveraging this structure will be an essential key to scaling up these concepts to use models in higher dimensions. \par
Several computational tools are being developed to compute invariant sets in state-space. For example, sums-of-squares polynomials can be cast as semi-definite programs (SOS programming), and have been used to compute robust funnels~\cite{majumdar2013robust} and robust regions of attraction~\cite{valmorbida2014roa_invariants}. Although these tools are typically used for analyzing convergence in the Lyapunov sense, it is reasonable to expect the same tools to work well for viable sets, which are also invariant sets. \par
Because these tools are computationally rather demanding, they have mostly been used for analysis. However, recent developments are pushing their utility for control, for example by combining SOS programming with reachability analysis in a hierarchical framework~\cite{singh2018robust}, or combining it with trajectory optimization~\cite{manchester2019robust}. Similar combinations with a view of viability in state-action space have, in my opinion, a lot of potential and offer exciting opportunities. \par
Directly learning a model of the dynamics while also learning a model of the safety measure is, in my opinion, one of the most promising yet straightforward next steps. In particular, each of these learning processes can benefit from the other in terms of sample efficiency: on the one hand, it is desirable to focus sampling in viable regions when gathering samples for learning dynamics. On the other hand, a model of the dynamics can be used to bootstrap learning the safety measure. Furthermore, predictions can help active sampling by 'checking ahead' to compare the safety of a state-action pair with the safety of the predicted next state. How to trade-off between uncertainty of the prediction compared to the model of the safety measure is something I hope to look into soon.