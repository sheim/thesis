% !TEX root = root.tex

\chapter{Applications of Viability in State Action Space}

We will do the following:

\section{Safe Model Free Learning}

\section{Quantify Robustness of Natural Dynamics}
Since any viable policy has to live in the viable set, we can compare systems with different natural dynamics. Define Natural Dynamics.
\subsection{Hierarchical Control, and templates and anchors}
We consider the natural dynamics of the underlying "blackbox".
\subsection{SLIP and NSLIP}
Connection to bifurcation analysis, and why this shows so much more. In state-space, viability kernel is exactly the same.
\subsection{Blackbox optimization of robustness}
Not prescriptive. Conclusion: slow (relies on brute force), but still useful for analysis, and because design can be done offline and it is embarrassingly parallel. Would benefit from improving scalability, and from approximations.
% \subsection{Analysis of damping in biomechanics}

\section{Safe, Model-free Learning}
We use this to formalize safe model-free learning. Collaboration with Alexander von Rohr and Sebastian Trimpe, who in particular brought in expertise in Gaussian processes and active sampling.
\subsection{Related work}
\subsection{Using the Measure as a Safety Function}
\subsection{Modeling the Measure as a Gaussian Process}
\subsection{Learning the Measure by Sampling}
\subsection{Results}
\subsection{Discussion and Outlook}

\section{Learning from outside the Viability Kernel}
We have seen from shaping that the reward landscape depends on all factors in the MDP.