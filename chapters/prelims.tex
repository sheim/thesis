% !TEX root = ../root.tex

\chapter{Preliminaries} \label{chap:prelims}
This chapter introduces relevant background information, to help readers from potentially different backgrounds to quickly gain enough knowledge to understand the context, relevance and importance of the contributions of each paper. Covered are the basics of legged locomotion, reinforcement learning and viability theory.

\section{Legged Locomotion}
FOCUS ONLY ON DYNAMIC LOCOMOTION
While most results in my publications are valid for dynamical systems in general, the motivation is grounded in legged locomotion. This field offers abundant natural inspiration, including ourselves. Legged systems also encapsulate many interesting challenges in dynamics: the dynamics of legged systems are non-smooth, due to impacts at foot touchdown. They are hybrid, as the governing equations of motion switch abruptly every time a foot touches down or lifts off. They are underactuated, due to the floating base. And of course they are typically also highly nonlinear. While passively stable legged systems exist, for most systems of interest to us they are passively unstable, and require feedback control.
On top of all this, the community of dynamic legged locomotion is both vibrant and friendly. In this section I will give a quick introduction to the field of legged locomotion.

\subsection{Reduced Order Models of Legged Locomotion}

Reduced order models have been a cornerstone to research in legged locomotion, allowing researchers to cope with the complexity of legged locomotion dynamics. \par
In this thesis, we will focus on the spring-loaded inverted pendulum (SLIP) model for running. This model originated in the biomechanics community to describe center-of-mass movement of running in humans~\cite{blickhan1989spring}. Comprising of a point-mass to represent the body and a massless spring to represent the leg, the model is parsimonious in its parameters, making it relatively easy to fit to data.
Several studies have then fit it to various animals of different sizes and with different numbers of legs~\cite{blickhan1993similarity,daley2006running,jindrich2002dynamic}. A two-legged extension also accurately predicts ground-reaction forces of both running and walking in humans~\cite{geyer2006compliant}, convincingly showing the importance of compliance in legged locomotion.
One of the benefits of these models, is that it gives a good sense of the system's \emph{natural dynamics}, or how the system naturally `wants' to move.
It is no surprise that, in parallel to the development of this model in the biomechanics community, Raibert developed his famous hopping robots using very similar models and intuition~\cite[See Figure 2.5]{raibert1986legged}.
There is a general consensus that running systems should feature dynamics with compliance, and a good control scheme will \emph{exploit the natural dynamics} of the system.
This is also observed in biology:~\textcite{daley2006running} used this model to explain the open-loop robustness to height perturbations observed in running birds.
Since this model is energy-conservative and neglects most degrees of freedom, it is often insufficient to describe many observations of interest. For example, extensions including actuation have been used to draw conclusions on control priorities~\cite{Birn-Jeffery3786,blum2014swing}.~\textcite{maus2015constructing} use a data-driven approach to suggest various extended state and control laws for the SLIP, based on experimental observations of humans. % Perhaps discuss this more \par

% They allow us to cope with the complexity of legged locomotion dynamics by capturing specific parts of the dynamics in simpler, low-dimensional models. They thus allow tractability and thorough analysis. \par
Reduced order models are also often used for mathematical analysis, as their low dimensionality allows for thorough investigation of the dynamics and parameters. For example,~\textcite{kuo2002energetics} analyzed `the simplest walking model', which allowed statements on the energetic benefits of toe-off to be made from simple, first-principles calculations.
In my own master thesis work~\cite{heim2016designing}, I derive the explicit equations of motion (\eom) of a running model with a tail. This is made possible by approximating the tail as a flywheel, and allows insight on scaling effects of different parameters of the tail, which could then be tested directly in hardware.
Reduced order models also allow numerical analysis by brute force, due to their low dimensionality. By making use of the \poincare section, the stability of a limit-cycle can be numerically computed by Floquet analysis~\cite{remy2011matlab}. Building on this approach, basins of attraction have been directly computed for a number of simple models~\cite{schwab2001basin,obayashi2016formation,cnops2015basin,rummel2008stable}, and in some cases also thorough bifurcations analysis~\cite{aoi2006bifurcation,merker2015stable,gan2018all}.~\textcite{byl2009metastable} used two simple models, the rimless wheel and the compass gait walker, to introduce a stochastic approach to stability, metastability, for walking systems. Metastability is a first step moving away from thinking about cyclic motion in terms of limit-cycles, by allowing the system to bounce around the limit-cycle without converging, and even diverging with probability one as time tends to infinity. Since the metastability analyses require substantial numbers of simulations, it would not be tractable on more sophisticated systems in higher dimensions. \par
% Reduced order models are also often used to explore and derive controllers. Indeed, often controllers for legged robots are split into hierarchies, with the higher levels of control reasoning on about a simpler, reduced order model as compared to the lower levels of control.
Perhaps the most successful example is the linear inverted pendulum (LIP) model, commonly used for the zero-moment point algorithm~\cite{kajita2001LIP,kajita2003ZMP}. Using the same and similar models,~\textcite{koolen2012capturability} compute $n$-step capture regions: regions of foot placement for which the system can come to a complete standstill within $n$ steps. Both these approaches can be applied by making use of a hierarchical controller: these simple models are used by a high-level planner. The output of this planner is then tracked by the low-level controller using a full-order model.
Another insightful result centers around the spring-loaded inverted pendulum (SLIP) model of running.~\textcite{wu20133} show an open-loop trajectory for the swing-leg which achieves deadbeat control: any ground-height perturbation can be completely rejected in a single step. Though only seldomly applied in practice~\cite{martin2017experimental}, showing the mere existence of such an open-loop controller is impressive.
Several other studies have evaluated different controllers for simple models by brute force~\cite{piovan2013two,cnops2015basin,piovan2015reachability}.
To make use of these results in a hierarchical fashion, the low-level controller would need to track the trajectories of a SLIP model. Substantial effort has been made in this direction~\cite{hutter2010slip, wensing2013high,poulakakis2009spring, renjewski2015exciting, martin2017experimental}. However, while this approach can exploit the natural dynamics of the simple model, it may neglect or even work against the natural dynamics of the true system. My opinion is the SLIP model is an excellent model with which to gain insight of what to look for, but not a good control objective. \par

% It can be challenging to make use of these controllers in practice since they reason about the discrete dynamics defined on the \poincare section. Unlike the hierarchical controllers mentioned above, these do not output a continuous trajectory to be tracked. Instead, 


\subsection{Control for Legged Locomotion}


% Dynamics of legged locomotion include many interesting properties that make them particularly challenging to control.  \par

A recurrent theme in legged locomotion is to design controllers that can exploit the natural dynamics of the system. Indeed, despite the very rich dynamics, it is possible to achieve locomotion with very simple approaches that exploit these natural dynamics.
Perhaps the most extreme examples are those inspired by passive dynamic walkers~\cite{mcgeer1990passive}. These robots feature minimal sensing and actuation, and are typically limited to flat ground~\cite{bhounsule2012design,wisse2006design}. They make for their lack in maneuverability with extreme efficiency: the controller mostly injects small amounts of energy and then allows the natural dynamics to take their own course.~\textcite{tedrake2005learning} saw in natural dynamics more than just the opportunity for efficient motion, and showed an example where the natural dynamics of a biped based on the passive dynamic walker could quickly and effectively learn and adapt an active controller. One of the key contributions of this thesis is to formalize this insight, and show how the intrinsic robustness of a system's natural dynamics helps learning control. \par
Many of the earlier successful legged robots relied on exploiting natural dynamics by using simple controllers in the form of clocks and oscillators~\cite{sprowitz2013towards,buchli2006resonance,altendorfer2001rhex,owaki2013simple}: motor positions are servoed along predetermined, periodic trajectories to the timing/phase of a clock/oscillator. All of these robots incorporate compliance in the form of mechanical springs. Once tuned to the natural dynamics, the clock and oscillator controllers generate very stable and dynamic gaits. With some feedback, the oscillators also adapt to the natural dynamics, resulting in different gaits depending on the morphology~\cite{owaki2013simple} and driving frequency~\cite{owaki2017quadruped}. % OWAKI oscillators with some feedback show adaptation to different natural dynamics
However, these approaches are often limited in versatility, and typically rely on a lot of intuition and trial-and-error to design. This also largely precluded their use in more unstable systems such as bipeds. \par
% Virtual Model Control
Optimal control offers a more explicit approach, in the form of
\begin{align*}
& \min_{u} J(x, u, t) \\
& \text{subject to } \phi(u, x, t) \leq 0,
\end{align*}
where $u$ is the control input, $x$ is the state of the system, $t$ is time, $J$ is an arbitrary, scalar cost function and $\phi$ is a vector of constraint functions.
The structure of the dynamics can be directly incorporated into the control law in the form of constraints, such that the control law directly and explicitly considers the natural dynamics of the system.
A designer is then allowed substantial freedom to design the overall behavior by modifying the cost function.
Solving this optimal control problem is computationally expensive. It is therefore common to solve these for trajectories of a reduced order model, such as the LIP model as mentioned earlier. More recently, centroidal momentum dynamics has gained popularity. It features additional dimensions which allow for more flexible trajectories~\cite{dai2014whole,koolen2016balance,ponton2016convex}, while still being dynamically consistent and computationally tractable.

Furthermore, most robots of interest are well described as rigid-body systems, and powerful and mature tools are readily available for modeling, identifying, synthesizing and computing controllers for these system.
% This will likely change in the future: robots with soft components~\cite{vonrohr2018gait,Buchler18ControlMusculo} or that are otherwise difficult to model accurately~\cite{surovik2018any} are starting to become more popular, and have some advantages compared to their more conventional counterparts. For the time being, 

% A designer is then allowed substantial freedom to design the overall behavior by modifying the cost function. As a result, optimal control has become one of the preeminent tools of the field, not only for control and planning~\cite{koolen2016design,ponton2016convex,winkler2018gait,mombaur_2009,deits2014footstep}, but also analysis and design~\cite{ha2018codesign,takahashi2019spring,mombaur_2009,Yesilevskiy_2018,Birn-Jeffery3786}.

\subsection{Hardware Design for Legged Locomotion}

Just as we focus on controllers that can exploit a system's natural dynamics, we will focus on hardware design that features beneficial natural dynamics.
Compliance has long been considered one of the more important features to design around.
% Even Raibert's first hopper~\cite{raibert1986legged} used a pneumatic actuator which also acted as an air-spring during stance. This approach requires a cumbersome pump.
We mentioned above several appraoches~\cite{sprowitz2013towards,buchli2006resonance,altendorfer2001rhex,owaki2013simple} which used highly-geared motors with and position control; the output of these `stiff' actuators then drove relatively soft springs, which provided the desired compliance. Among the benefits of this approach is that they allow very simple controllers to achieve remarkable stability, while running on cheap hardware with low update rates (on the order of 10-100 Hz). The soft springs are also typically placed distally, and therefore minimize unsprung mass and protect the motors from harsh impacts. However, the complexity is traded off into the hardware design, and tuning hardware is usually much more time-consuming and expensive than tuning software. Furthermore, once the built, the parameters cannot be changed without disassembling the robot. This limits the robot's versatility, as different behaviors may require vastly different natural dynamics. \par
A different approach has been to incorporate the compliance directly in the actuator. Indeed, Raibert's first hoppers used pneumatic actuators, which act as air-springs in addition to injecting power. They require, however, cumbersome pumps which are typically heavy and inefficient.
A popular solution are series-elastic actuators (SEA)~\cite{pratt1995series}. In this case, the output of a highly-geared motors is coupled in series with a relatively stiff spring. By measuring the position of both ends of the spring, the force output at the end-effector can be trivially calculated. This allows force/torque control at the joints, and compliance of relatively arbitrary form can be generated, within the limitations of the motor constraints and kinematics. Several of the most successful robots of today are built with this type of actuation, such as ANYMAL~\cite{hutter2016anymal} and Cassie\footnote{While there are no publications on the hardware design of Cassie, this is both clear from looking at the robot and is confirmed from employees of Agility Robotics.}.
\par
A third approach eschews all mechanical springs in favor of generating compliance purely through actuation. This requires low gear-ratios in order to minimize reflected inertia and achieve transparency, while still maintaining high peak torque outputs. This results in `pancake-style' motor designs SANGBAE.
Fortunately, the widespread success of quadcopters has dramatically reduced the cost of off-the-shelf outrunner motors. Together with field-oriented control, sometimes called vector control, these motors provide a cheap solution at smaller scales. Many small and medium-sized robots capable of highly dynamic motion and direct torque-control are emerging, such as the MIT mini-cheetah~\cite{katz2019mini}, the Minitaur~\cite{kenneally2016design} and MPI's own Solo~\cite{grimminger2019open}. Since mass scales roughly cubically\footnote{Mass will scale cubically to length if we assume isometric scaling and constant, uniform density. These assumptions do not typically hold for robot designs, but do hold to some reasonable degree.}, these smaller scale robots are mechanically sturdy~\cite{biewener2005biomechanical}, and can operate at torques that are much safer to handle. This makes them excellent experimental platforms, especially for learning control, since failures may be more common.

% Design of legged robots has come a long way in the last decade. The overall trend is to enable torque control at the joints.

% Scale: we want small robots

% Types of actuation: we want torque-control. This led to hydraulic ATLAS and HyQ, then series-elastic actuation ANYMAL, Cassie. Inspired by the haptics community, SANGBAE proprioceptive motors. Low-gear ratio allows high transparency. Commercial success of quadcopters has lowered the cost of small and mid-sized outrunner motors, allowing a new generation of small and cheap robots with torque control at the joints CITATIONS.

% This new generation of small and cheap robots the ones that most interest us. As research platforms they will allow experimentation on hardware, including model-free learning. The small size lends an inherent sturdiness to failures CITATION. The cheap production allows them to be repaired quickly and easily. 

% \subsection{Reduced-order Models of Running}

% A major challenge for these tools is still scalability. In high-dimensional systems, optimal control is often limited to finding locally optimal solutions. Alternatively, a popular approach is to split control into hierarchies. A high-level controller reasons on a reduced-order model, which becomes tractable. These solutions are then mapped to the full state-space~\cite{herzog2016momentum,dai2014whole,wensing2013high}. These reduced-order models are also particularly convenient for analysis: it is easier for us to reason about low-dimensional systems that we can easily visualize. Furthermore, they can amenable to more powerful but less tractable tools, including brute-force. \par

% It is important at this point to make a distinction between models that are useful for control and models that are useful for analysis. For control, a reduced-order model should be easy to map back into the full-order model, and provide provable guarantees. For example, the LIP assumes a constant height for walking, which does not describe natural walking well. It does however make computation tractable (convex?) and can provide guarantees, making it a good choice for control. A model for analysis should be easy to understand, and closely match the qualitative properties of interest. For example, the spring-loaded inverted pendulum (SLIP) provides very good intuitive understanding for the compliant properties of legs, both in running animals~\cite{blickhan1989spring,rummel2008stable,jindrich2002dynamic}, as well as in some robots~\cite{raibert1986legged,altendorfer2004stability}. They have also been used to conceptually explore various control concepts, such as open-loop deadbeat control~\cite{wu20133,palmer2014periodic} or reachability~\cite{piovan2015reachability}. \par
% While many efforts have been made to map these directly to the full-order model~\cite{wensing2013high,hutter2010slip,poulakakis2009spring}, typically this step requires substantial effort in terms of implementation and computation. The resulting control is typically brittle to model inaccuracies. Furthermore, forcing the high-dimensional dynamics of the actual system to behave on the low-dimensional manifold of the SLIP can potentially override the true natural dynamics. For these reasons, we recommend find the value of these models in their \emph{descriptive} role. We do not recommend their \emph{prescriptive} use to design formal controllers. \par
% Most of my work is built around the analysis of the SLIP model, since it can be reduced to a 1-dimensional state-space and 1-dimensional action-space. I would like to emphasize that we choose to use this as a \emph{descriptive} model, that is, for analysis. I do not recommend directly using it as a prescriptive model, that is, to design controllers assuming the system will behave exactly like a SLIP model. \par

\section{Learning Control}

The recent success in machine learning, in particular with reinforcement learning (RL), is pushing another alternative to conventional model-based optimal control. The work of my phd has been heavily influenced by ideas from reinforcement learning, especially the concept of state-action space and model-free control. For readers less familiar with this field, I will give a brief introduction of the concepts that are useful for understanding my work.

\subsection{Model-free and model-based}
At its core, reinforcement learning is simply model-free optimal control. Just as dynamic programming and value iteration, the standard RL algorithm is based on Bellman updates. 

Let us start with standard reinforcement learning. At its core, it is simply a model-free version of optimal control. EXAMPLE WITH VALUE ITERATION AND Q-LEARNING.
At the end of the day, all just Bellman updates. \par

While early efforts in reinforcement learning emphasized the strength of being purely model-free, it has since become clear that models are indeed useful. Although model-predictions are never perfectly accurate, they are usually still quite good, especially for the systems we are interested in. A lot of more recent effort in learning control mixes model-free and model-based tools. \par

Learn a model, then use it (sys ID).

Start with model-based tools, then learn an additional additive controller to account for un-modeled dynamics.

Use a model to generate data (simulation).

\subsection{Shaping}

In RL, shaping refers to shaping the reward-landscape, in order to make it easier to for a learning agent to learn on.
Most of the effort in this field is called curriculum design: the agent learns on easier tasks which inform it about the original policy.
There can also be shaping by modifying the reward function. This can be a temporary, handcrafted change, or it can be IRL.
It is akin to solving a convex approximation of the cost-function in model-based optimization. As RL often takes a model-free approach, usually there is no emphasis on lower and upper bounds (as there is in model-based optimization).
Mostly focused on changing $R$ in the MDP, in some cases a combination of $R$ and $P$.

% \section{Template Models}
% To explore 

\section{Viability and Backreachability}

The core of my work is based on viability theory, first pioneered by~\textcite{aubin2011viability}. The development of viability theory is largely motivated by observations of dynamical systems in nature and society, whose behavior somehow avoids chaos or pure randomness, yet never seem to settle at a resting state, an equilibrium.
Simple examples are Darwinian evolution, economics or politics of state. There is no terminal equilibrium state for these systems, or at least, none that we can identify or foresee.
The classical mathematical tools based on convergence to such an equilibrium state are therefore ill-suited to describe them, although many have found ways to adapt them to the task, for example by assuming a time-varying equilibrium state, which the system chases but never reaches. \par
Viability theory provides are more appropriate and direct description. First, sets of failure states are defined, which the system must be able to avoid. From this naturally emerges the \emph{viability kernel}, the maximal set of states from which there exist control inputs which keep the system inside the viability kernel, under the constraint of never entering the set of failure states. In simpler terms, if the system ever leaves the viability kernel, it means it can no longer return inside of it, and is doomed to eventually enter the failure set.
The viability kernel thus includes all possible regions (or "basins") of attraction, since these regions avoid the failure set by their quality of converging to a non-failing state. However, it does not require the property of convergence convergence, and the system is free to roam the viability kernel, even stochastically, as long as it never leaves this set. \par
This provides us with a powerful way of thinking, which is particularly appropriate for considering learning systems. Indeed, although the final goal in learning control may be a control policy which generates a large region of attraction with strong convergence properties, these properties typically cannot be relied on during the learning process. \par
Viability theory therefore allows us to make very general statements of a dynamical system's behavior. However, this comes at a price: computing viable sets is typically computationally expensive, and for the types of dynamics we are most interested in it often relies on brute force. A related approach is the computation of back-reachable sets. The back-reachable set is the set of all states from which the system can reach a specified target set. This differs from convergence to equilbria, for two important distinctions. First, it does not assume convergence towards the target set, only the existence of convergence. Second, the target set does not necessarily include equilibria, and the system may not necessarily be able to remain inside the target set after reaching it. In practice, viability kernels and the back-reachable sets are largely interchangeable. For many cases of interest, they happen to coincide. Indeed, the viability kernel is in essence the back-reachable set of itself. For an introduction to methods for computing these sets, we recommend \cite{bansal2017hamilton,liniger2017real}.