% !TEX root = ../root.tex

\chapter{Published Work} \label{chap:pubs}
This cumulative thesis is based on the four first author, peer-reviewed publications I produced during my phd studies. In this chapter, I give a recommended reading order, abstract and commentary. Then, my individual contribution for each publication is clarified, as well as some context for each venue of publication.

\section{Summaries}
The recommended reading order is:
\begin{enumerate}
    \item Shaping in Practice: Training Wheels to Learn Fast Hopping Directly in Hardware
    \item Beyond Basins of Attraction: Quantifying Robustness of Natural Dynamics
    \item A Learnable Safety Measure
    \item Learning from Outside the Viability Kernel: Why we should build Robots that can Fall with Grace
\end{enumerate}
The first paper is an empirical study which captures much of the motivation. The second paper formalizes viable sets in state-action space, which is the core theoretical contribution of the thesis. The third paper builds on the second to formalize model-free safe learning. The fourth paper is an offshoot of the second, and points out a minor, counter-intuitive result.

\subsection{Shaping in Practice}
\textbf{Abstract: }
Learning instead of designing robot controllers can greatly reduce engineering effort required, while also emphasizing robustness. Despite considerable progress in simulation, applying learning directly in hardware is still challenging, in part due to the necessity to explore potentially unstable parameters. We explore the of concept shaping the reward landscape with \emph{training wheels}; temporary modifications of the physical hardware that facilitate learning. We demonstrate the concept with a robot leg mounted on a boom learning to hop fast. This proof of concept embodies typical challenges such as instability and contact, while being simple enough to empirically map out and visualize the reward landscape. Based on our results we propose three criteria for designing effective training wheels for learning in robotics. \par
\textbf{Commentary: }
This empirical study served to understand what are the most important aspects when considering learning in hardware. One of the most important take-aways is that failures are common and typically provide very little useful information to learn from. When considering safety in robot learning, the usual motivations are avoiding damage (to the robot or the world) and time-consuming resets and repairs. Here we see, however, that it is also important in order to learn more reliably. \par
% * description: a simple and low-dimensional system. An open-loop sinusoidal controller parameterization with a spring-loaded leg, a combination which is intuitively known to be very robust. The low-dimensionality allows us to quickly map out the actual reward landscape by sampling a grid empirically. We can therefore obtain ground-truth for the system and compare how the reward landscape changes as we adjust a physical parameter (in our case the total mass of the system).
% We can thus show some important aspects in a simple and direct manner.
% First, large patches of the 

% To keep the focus of the study on the effect of changing morphology, all other aspects are simplified as much as possible. The robot used is a single two degree-of-freedom (DoF) leg, constrained to 2D motion by a boom. Only the hip joint is actuated, where and a passive spring-loaded ankle joint, similar to the hopper used
% Specifically, we empirically map out the reward landscape while changing a physical parameter in a simple robot system. We then show 
\textbf{Venue: }the International Conference for Robotics and Automation (ICRA) is the largest and one of the most important conferences in robotics, alongside the International Conference on Intelligent Robots and Systems (IROS). Publications in these venues often have major impact, and they are considered a premier publication venue for the field of robotics. However, variance on the quality of publications is also high, presumably due to high variance in reviewing quality.

\subsection{Beyond Basins of Attraction}
\textbf{Abstract: }
Properly designing a system to exhibit favorable natural dynamics can greatly simplify designing or learning the control policy. However, it is still unclear what constitutes favorable natural dynamics and how to quantify its effect. Most studies of simple walking and running models have focused on the basins of attraction of passive limit-cycles and the notion of self-stability. We instead emphasize the importance of stepping beyond basins of attraction. We show an approach based on viability theory to quantify robust sets in state-action space. These sets are valid for the family of all robust control policies, which allows us to quantify the robustness inherent to the natural dynamics before designing the control policy or specifying a control objective.
We illustrate our formulation using spring-mass models, simple low dimensional models of running systems. We then show an example application by optimizing robustness of a simulated planar monoped, using a gradient-free optimization scheme. Both case studies result in a nonlinear effective stiffness providing more robustness. \par
\textbf{Commentary: }
To the best of my knowledge, this is the first formalization of how morphology or mechanical design influences robustness. The formalization of viable sets in state-action space allow, for the first time, quantified comparisons of different designs in terms of robustness, without making assumptions on the control policy. Furthermore, focusing on robustness in terms of avoiding failure instead of convergence is a subtle but important change in how we reason about locomotion. Although the fundamental mathematical objects are defined in this paper, there is a lot of room for future work in applying these objects in different contexts and for specific systems. Another open topic are computational tools to apply this in practice. In particular, scaling to higher dimensions remains and important open challenge. % rewrite

\textbf{Venue: }
Transactions on Robotics is the flagship journal of the IEEE Robotics and Automation Society. It is widely considered one of the top journals in robotics, either on par or a close second to the International Journal on Robotics Research (IJRR).

\subsection{A Learnable Safety Measure}
\textbf{Abstract: }
Failures are challenging for learning to control physical systems since they risk damage, time-consuming resets, and often provide little gradient information. Adding safety constraints to exploration typically requires a lot of prior knowledge and domain expertise. We present a safety measure which implicitly captures how the system dynamics relate to a set of failure states. Not only can this measure be used as a safety function, but also to directly compute the set of safe state-action pairs. Further, we show a model-free approach to learn this measure by active sampling using Gaussian processes. While safety can only be guaranteed after learning the safety measure, we show that failures can already be greatly reduced by using the estimated measure during learning. \par
\textbf{Commentary: }
\textbf{Venue: }
the Conference on Robot Learning (CoRL) is a young conference: this publication appeared in its third edition. It is loosely modeled after the Robotics Science and Systems (RSS) conference, a small, single-track conference known for its high-quality and low acceptance rate. CoRL has so far managed to maintain these standards. In its third edition, 112 papers were selected out of 398 submissions. Papers in the second edition have an average of X citations after a year. This high citation count can be partly attributed to the high standards of the conference, and partly to the current popularity (and therefore high publication rate) of machine learning. \par

\textbf{Contribution: }



\subsection{Learning from Outside the Viability Kernel}
\textbf{Abstract: }
Despite impressive results using reinforcement learning to solve complex problems from scratch, in robotics this has still been largely limited to model-based learning with very informative reward functions. One of the major challenges is that the reward landscape often has large patches with no gradient, making it difficult to sample gradients effectively. We show here that the robot state-initialization can have a more important effect on the reward landscape than is generally expected. In particular, we show the counter-intuitive benefit of including initializations that are \emph{unviable}, in other words initializing in states that are doomed to fail.
\textbf{Commentary: }
\textbf{Venue: }

\section{Individual Contributions}
% turn this into a table
\begin{enumerate}
    \item Training wheels
    \item Beyond Basins
    \item Safety Measure
    \item Unviable
\end{enumerate}

\subsection{Training Wheels}


