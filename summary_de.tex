% !TEX root = root.tex

\chapter*{Zusammenfassung}
\addcontentsline{toc}{chapter}{Zusammenfassung}
\emph{Wie k\"{o}nnen wir Robotern erm\"{o}glichen, modellfrei und direkt auf der Hardware zu lernen?} \par

Das maschinelle Lernen nimmt als Standardwerkzeug im Arsenal des Robotikers seinen Platz ein. Es gibt jedoch einige offene Fragen, wie man die Kontrolle über physikalische Systeme lernen kann. Diese Arbeit gibt zwei Antworten auf diese motivierende Frage. \par
Das erste ist ein formales Mittel, um die inh\"{a}rente Robustheit eines gegebenen Systemdesigns zu quantifizieren, bevor der Controller oder das Lernverfahren entworfen wird. Dies unterstreicht die Notwendigkeit, sowohl das Hard- als auch das Software-Design eines Roboters zu berücksichtigen, da beide Aspekte in der Systemdynamik untrennbar miteinander verbunden sind. \par
Die zweite ist die Formalisierung einer Sicherheitsma{\ss}, die modellfrei erlernt werden kann. Intuitiv zeigt diese Ma{\ss} an, wie leicht ein Roboter Fehlschl\"{a}ge vermeiden kann. Auf diese Weise k\"{o}nnen Roboter unbekannte Umgebungen erkunden und gleichzeitig Ausf\"{a}lle vermeiden. \par
Die wichtigsten Beitr\"{a}ge dieser Dissertation basieren sich auf der Viabilit\"{a}tstheorie. Viabilit\"{a}t bietet eine alternative Sichtweise auf dynamische Systeme: Anstatt sich auf die Konvergenzeigenschaften eines Systems in Richtung Gleichgewichte zu konzentrieren, wird der Fokus auf Menge von Fehlerzust\"{a}nden und die F\"{a}higkeit des Systems, diese zu vermeiden, verlagert. Diese Sichtweise eignet sich besonders gut für das Studium der Lernkontrolle an Robotern, da Stabilit\"{a}t im Sinne einer Konvergenz w\"{a}hrend des Lernprozesses selten gew\"{a}hrleistet werden kann. \par
Der Begriff der Viabilit\"{a}t wird formal auf den Zustand-Aktion-Raum erweitert, mit Viabilit\"{a}tsmengen von Staat-Aktionspaaren. Eine über diese Mengen definierte Ma{\ss} erm\"{o}glicht eine quantifizierte Bewertung der Robustheit, die für die Familie aller fehlervermeidenden Regler gilt, und ebnet den Weg für ein sicheres, modellfreies Lernen. \par
Die Arbeit beinhaltet auch zwei kleinere Beitr\"{a}ge. Der erste kleine Beitrag ist eine empirische Demonstration der Shaping durch ausschlie{\ss}liche Modifikation der Systemdynamik. Diese Demonstration verdeutlicht die Bedeutung der Robustheit gegenüber Fehlern für die Lernkontrolle: Ausf\"{a}lle k\"{o}nnen nicht nur Sch\"{a}den verursachen, sondern liefern in der Regel auch keine nützlichen Gradienteninformationen für den Lernprozess. \par
Der zweite kleine Beitrag ist eine Studie über die Wahl der Zustandsinitialisierungen. Entgegen der Intuition und der üblichen Praxis zeigt diese Studie, dass es zuverl\"{a}ssiger sein kann, das System gelegentlich aus einem Zustand zu initialisieren, der bekannterma{\ss}en unkontrollierbar ist. 